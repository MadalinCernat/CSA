
\documentclass{article} 

\usepackage{geometry} 
\usepackage{amsmath} 
\usepackage{graphicx} 
\usepackage{xlop}

\newcommand{\question}[2][]{\begin{flushleft}
        \textbf{Question #1}: \textit{#2}


\end{flushleft}}

\newcommand{\sol}{\textbf{Solution}:} 
\newcommand{\maketitletwo}[2][]{\begin{center}
        \Large{\textbf{Homework 1}
            
            CSA}
        \vspace{5pt}
        
        \normalsize{Mădălin Cernat
        
        \today}     
        \vspace{15pt}
        
\end{center}}

\begin{document}
    \maketitletwo[5]
    
    	\question[1]{Convert the following numbers from base 10 to 2 and then to 16:} 
	\begin{gather*}
	  4_{(10)} \implies  0100_{(2)} \implies 4_{(16)}\\
	  10_{(10)} = 8 + 2 = 2^3 + 2^1 \implies 1010_{(2)} \implies A_{(16)}\\
	  15_{(10)} = 8 + 4 + 2 + 1 = 2^3 + 2^2 + 2^1 + 2^0 \implies 1111_{(2)} \implies F_{(16)}\\
	  32_{(10)} = 2^5 \implies 0010\ 0000_{(2)} \implies 20_{(16)}
	\end{gather*}
    
    	\question[2]{Convert the following numbers from base 10 to 16 and then to 2:}
   	\begin{gather*}
		3_{(10)} \implies 3_{(16)} \implies 0011_{(2)}\\
		11_{(10)} \implies B_{(16)} \implies 1011_{(2)}\\
		16_{(10)} \implies 10_{(16)} \implies 10000_{(2)}\\
		17_{(10)} \implies 11_{(16)} \implies 10001_{(2)}\\
	\end{gather*}
    
    	\question[3]{Convert the following numbers from base 2 to 16:}
	\begin{gather*}
		1010_{(2)} \implies A_{(16)}\\
		0111_{(2)} \implies 7_{(16)}\\
		1111_{(2)} \implies F_{(16)}\\
		1000\ 1010_{(2)} \implies 8A_{(16)}\\
		0001\ 1010\ 1111_{(2)} \implies 1AF_{(16)}\\
	\end{gather*}
	\newpage

	\question[4]{Convert the following numbers from base 16 to 2:}
	
	\begin{gather*}
		3_{(16)} \implies 0011_{(2)}\\
		A_{(16)} \implies 1010_{(2)}\\
		F_{(16)} \implies 15_{(10)} \implies 1111_{(2)}\\
		2B_{(16)} \implies 0010\ 1011_{(2)}\\
		2F8_{(16)} \implies 0010\ 1111\ 1000_{(2)}\\
	\end{gather*}

	\question[5]{Compute the following expressions directly in base 2 (without converting to base 10):}

	\begin{gather*}
		01_{(2)} + 01_{(2)} = 10_{(2)}\\
		10_{(2)} + 10_{(2)} = 100_{(2)}\\
		111_{(2)} + 001_{(2)} = 1000_{(2)}\\
		1010_{(2)} + 0001_{(2)} =  1001_{(2)}\\
		1000_{(2)} - 10_{(2)} = 0110_{(2)}\\
	\end{gather*}	

	\question[6]{Compute the following expressions directly in base 16 (without converting to base 10):}

	\begin{gather*}
		9_{(16)} + 1_{(16)} = A_{(16)}\\
		B_{(16)} + 2_{(16)} = D_{(16)}\\
		F_{(16)} + 1_{(16)} = 10_{(16)}\\
		10_{(16)} + A_{(16)} = 1A_{(16)}\\
		10_{(16)} - 2_{(16)} = E_{(16)}\\
		B_{(16)} - 3_{(16)} = 8_{(16)}\\
	\end{gather*}	

	\question[7]{Check, using at least two of the complementary code rules, if:}
		\begin{itemize}
			\item 9A7D(16) and 7583(16) are complementary in a location of 2 bytes
			\begin{equation*}
				\begin{aligned}
					&\left.\begin{aligned}
						9A7D_{(16)} = 1001\ 1010\ 0111\ 1101_{(2)} \\
						7583_{(16)} = 0111\ 0101\ 1000\ 0011_{(2)} \\ 
					\end{aligned}\right\}\Rightarrow\text{They are \textbf{not complementary} in a location of 2 bytes.}
				\end{aligned}
			\end{equation*}
			
			\begin{equation*}
				10000_{(16)} - 09A7D_{(16)} = 	6583_{(16)} \neq 7583_{(16)}
			\end{equation*}

				
			\item 000F095D(16) and FFF0F6A3(16) are complementary in a location of 4 bytes
			\begin{equation*}
				10000000_{(16)} - 000F095D_{(16)} = 1FFF0F6A3_{(16)} 
			\end{equation*}
			So they are \textbf{complementary} in a location of 4 bytes.

			\item 4BA1(16)  and 5C93(16) are complementary in a location of 2 bytes
			\begin{equation*}
				\begin{aligned}
					&\left.\begin{aligned}
						 4BA1_{(16)} = 0100\ 1011\ 1010\ 0001_{(2)} \\
						 6C93_{(16)} = 0110\ 1100\ 1001\ 0011_{(2)} \\
					\end{aligned}\right\}\Rightarrow\text{They are \textbf{not complementary} in a location of 2 bytes.}
				\end{aligned}
			\end{equation*}
			
			\item 7F(16) and 81(16) are complementary in a location of 1 byte
			\begin{equation*}
				7F_{(16)} = 0111\ 1111_{(2)} \\
			\end{equation*}
				We invert the bits starting at the first 1 from right to left, to obtain the complementary: \begin{equation*} 1000\ 0001_{(2)}. \end{equation*}
			\begin{equation*}
				1000\ 0001_{(2)} = 81_{(16)}
			\end{equation*}
				So the bytes are \textbf{complementary}.

			\item 732A(16)  and 4E58(16) are complementary in a location of 2 bytes
			\begin{equation*}
				732A_{(16)} = 0111\ 0011\ 0010\ 1010_{(2)} \implies 1000\ 1100\ 1101\ 0110_{(2)} = 8CD6_{(16)} \neq 4E58_{(16)}
			\end{equation*}
		\end{itemize}



	\question[8]{Write the 8-bits unsigned representation for the following numbers:}
	\begin{gather*}
		8_{(10)} = 0000\ 1000_{(2)} \\
		67 = 2^6 + 2^1 + 2^0 = 0100\ 0011_{(2)} \\
		230 = 2^7 + 2^6 + 2^5 + 2^2 + 2^1 = 1110\ 0110_{(2)} \\
	\end{gather*}

	\question[9]{Write the 16-bits signed representation for the following numbers:}
	\begin{gather*}
		6_{(10)} = 0000\ 0110_{(2)} \implies -6_{(2)} = 1111\ 1010_{(2)} \\
		121_{(10)} = 0111\ 1001_{(2)} \implies -121_{(2)} = 1000\ 0111_{(2)} \\
		70_{(10)} = 0100\ 0110_{(2)}
	\end{gather*}

\end{document}